\documentclass{article}
\usepackage[utf8]{inputenc}
\usepackage{amsmath}
\usepackage{amssymb}
\usepackage[dvipsnames]{xcolor}
\usepackage[a4paper, total={6in, 10in}]{geometry}
\usepackage{tcolorbox}
\usepackage{mdframed}
\usepackage[hidelinks]{hyperref}
\usepackage{amsfonts}
\usepackage{mathrsfs}
\usepackage{centernot}
\usepackage{graphicx}
\graphicspath{ {./images/} }
\usepackage[percent]{overpic}

\setlength{\jot}{10pt}

\newmdenv[
  topline=false,
  bottomline=false,
  skipabove=\topsep,
  skipbelow=\topsep,
  leftmargin=-10pt,
  rightmargin=-10pt,
  innertopmargin=0pt,
  innerbottommargin=0pt,
  linecolor=blue
]{siderules}

\title{Automata theory and its applications\\Notes}
\author{toshinari tong}
\begin{document}
\maketitle
\section{Finite Automata}
\subsection{Finite Automata}
\textbf{omega}: \(\omega=\{0,1,2,...\}\)\\
\textbf{alphabet}: \(\Sigma=\{a_{1},...,a_{n}\}\)\\
\textbf{input}: \(u=\sigma_{1}\sigma_{2}...\sigma_{m}\) where \(\sigma_{i}\in\Sigma\)\\
\textbf{empty input}: \(\lambda\)\\
\textbf{finite words}: \(\Sigma^{*}=\{\sigma_{1}\sigma_{2}...\sigma_{m}|\sigma_{1},\sigma_{2},...,\sigma_{m}\in\Sigma,m\in\omega\}\)\\
\textbf{\(\Sigma\)-language}: any subset of \(\Sigma^{*}\)\\
\textbf{set of states}: \(S=\{s_{0},...,s_{k}\}\)\\
\textbf{sequentiality}: the next state of the system is one of the states determined by the pair \((s,\sigma)\)\\
\\
\textbf{nondeterministic finite automaton}: \(\mathcal{A}=(S,T,I,F)\) where\\
\null\qquad \(S\) is a finite nonempty set called the set of states\\
\null\qquad \(T\subseteq S\times\Sigma\times S\) is a nonempty set called the transition table\\
\null\qquad \(I\subseteq S\) is called the set of initial states\\
\null\qquad \(F\subseteq S\) is called the set of final states\\
Finite automata are the simplest mathematical abstractions of \textbf{discrete sequential systems}.\\
\(T(s,\sigma)=\{s'\in S|(s,\sigma,s')\in T\}\)\\
\textbf{deterministic automaton}: \(\mathcal{A}\) is deterministic if \(|I|=1\) and \(\forall s\in S,\forall\sigma\in\Sigma,|T(s,\sigma)|=1\).\\
\\
\textbf{computation / run}: a sequence of states of length \(m+1\) from given input of length \(m\):\\
\null\qquad stage 1: \(\mathcal{A}\) choose an initial state \(s_{1}\in I\)\\
\null\qquad stage \(n\): suppose sequence \(s_{1}s_{2}...s_{n-1}\) is produced. the automaton chooses an \(s_{n}\in S\) s.t. \((s_{n-1},\sigma_{n-1},s_{n})\in T\)\\
If \(\mathcal{A}\) is deterministic, then it has a computation on any input and this computation is unique.\\
\textbf{successful computation / \(\mathcal{A}\) accepts an input}: \(s_{m+1}\in F\)\\
\\
\textbf{transition diagram}: this automaton accepts words \((ab)^{n}aa\)\\
\includegraphics{auto1}\\
\\
\textbf{accessibility}: \(G_{0}(s)=\{s\}\), \(G_{1}(s)=\{s_{1}|\exists\sigma\in\Sigma, (s,\sigma,s_{1})\in T\}\cup G_{0}(s)\),\(G_{n+1}(s)=G_{n}(s)\cup\bigcup_{s_{1}\in G_{n}(s)}G_{1}(s_{1})\)\\
\textbf{accessibility operator}: \(G(s)=\bigcup_{n}G_{n}(s)\)\\
\\
\textbf{finite automaton (FA) recognizable language}: \(L\subseteq\Sigma^{*}\) s.t. \(\exists\mathcal{A},L=L(\mathcal{A})=\{w\in\Sigma^{*}|\mathcal{A} \text{ accepts }w\}\)\\
\begin{siderules}\color{blue}\textit{EXERCISE 2.2.7}\color{black}\\\\
\color{blue}1. Give an example of a nfa with 4 states over the alphabet \(\{a,b\}\) s.t. \(\mathcal{A}\) recognizes language \(L\) containing exactly those \(w\) that have \(a\) at the 3rd position from the right of \(w\).\\\color{black}
\includegraphics[width=7cm]{auto2}\\
\color{blue}2. Give an example of a deterministic automaton that recognizes the languatge \(L\) defined in the previous item.\color{black}\\
\includegraphics[width=7cm]{auto3}
\end{siderules}
\subsection{Closure Properties}
\color{red}Let \(\mathcal{A}_{1}\) and \(\mathcal{A}_{2}\) be automata. Then there exists an automaton \(\mathcal{A}\) s.t. \(L(\mathcal{A})=L(\mathcal{A}_{1})\cup L(\mathcal{A}_{2})\).\color{black}\\
First suppose \(S_{1}\cap S_{2}=\varnothing\). If they have common states, we can just rename elements of \(S_{2}\) and create a new copy of \(\mathcal{A}_{2}\). Define \(\mathcal{A}=(S,T,I,F)\) as follows:\\
\null\qquad 1. \(S=S_{1}\cup S_{2}\).\\
\null\qquad 2. \(T=T_{1}\cup T_{2}\).\\
\null\qquad 3. \(I=I_{1}\cup I_{2}\).\\
\null\qquad 4. \(F=F_{1}\cup F_{2}\).\\
Denote \(\mathcal{A}=\mathcal{A}_{1}\oplus\mathcal{A}_{2}\).\\\\
\color{red}Let \(\mathcal{A}_{1}\) and \(\mathcal{A}_{2}\) be automata. Then there exists an automaton \(\mathcal{A}\) s.t. \(L(\mathcal{A})=L(\mathcal{A}_{1})\cap L(\mathcal{A}_{2})\).\color{black}\\
\null\qquad 1. \(S=S_{1}\times S_{2}\).\\
\null\qquad 2. \(T=\{((s_{1},s_{2}),\sigma,(t_{1},t_{2}))\;|\;(s_{1},\sigma,t_{1})\in T_{1},(s_{2},\sigma,t_{2})\in T_{2}\}\).\\
\null\qquad 3. \(I=I_{1}\times I_{2}\).\\
\null\qquad 4. \(F=F_{1}\times F_{2}\).\\
Denote \(\mathcal{A}=\mathcal{A}_{1}\times\mathcal{A}_{2}\).\\\\
\begin{siderules}\color{blue}\textit{EXERCISE 2.3.2}\color{black}\\\\
\color{blue}Let \(\mathcal{A}_{1}\) and \(\mathcal{A}_{2}\) be finite automata. Construct an automaton which
accepts the union \(L(\mathcal{A}_{1})\cup L(\mathcal{A}_{2})\) and whose set of states is \(S_{1}\times S_{2}\) and the transition table is the one for \(\mathcal{A}_{1}\times\mathcal{A}_{2}\).\color{black}\\\\
\(F=\{(f_{1},x_{2})\;|\;f_{1}\in F_{1},x_{2}\in S_{2}\}\cup \{(x_{1},f_{2})\;|\;f_{2}\in F_{2},x_{1}\in S_{1}\}\)
\end{siderules}
\begin{siderules}\color{blue}\textit{EXERCISE 2.3.3}\color{black}\\\\
    \color{blue}Consider the 2 automata. Construct the automaton that accepts the intersection.\\\\\color{black}
\end{siderules}
\color{red}If \(\mathcal{A}\) is a deterministic automaton, then \(\mathcal{A}^{c}=(S,T,I,S\backslash F)\) accepts the complement of \(L(\mathcal{A})\), that is \(L(\mathcal{A}^{c})=\Sigma^{*}\backslash L(\mathcal{A})\).\color{black}\\\\
However if the original automaton is not deterministic, then this doesn't work: Even if there is a computation \(s_{1},s_{2},...,s_{m+1}\) s.t. \(s_{m+1}\notin F\),
this does not guarantee that \(u\) is not accepted by \(\mathcal{A}\). For example, maybe choosing some other initial state or some other path may result in a successful run. 
In other words, \(u\) is not accepted iff all computations of \(\mathcal{A}\) on \(u\) are unsuccessful.\\\\
The \textbf{power automaton} \(\mathcal{A}^{d}=(S^{d},T^{d},I^{d},F^{d})\) is defined as follows:\\
\null\qquad 1. \(S^{d}=\mathcal{P}(S)\)\\
\null\qquad 2. \(T^{d}=\{(S_{1},\sigma,S_{2})\;|\; S_{2}=\{s'\;|\;\exists s\in S_{1} \text{ s.t. } (s,\sigma,s')\in T\}\}\)\\
\null\qquad 3. \(I^{d}=\{I\}\)\\
\null\qquad 4. \(F^{d}=\{G\subseteq S\;|; G\cap F\neq\varnothing\}\) if \(F\neq\varnothing\).\\
\includegraphics{auto4}
\includegraphics{auto5}\\
\color{red}Let \(\mathcal{A}\) be a nfa. Then the power automaton \(\mathcal{A}^{d}\) is deterministic and \(L(\mathcal{A})=L(\mathcal{A}^{d})\).\color{black}\\
If \(u=\sigma_{1}\sigma_{2}...\sigma_{m}\) is accepted by \(\mathcal{A}\), then by definition of \(F^{d}\) it is accepted by \(\mathcal{A}^{d}\).\\
If \(u\) is accepted by \(\mathcal{A}^{d}\), let \(S_{1},S_{2},...,S_{m+1}\) be a computation of \(\mathcal{A}^{d}\) on \(u\); then there exists an \(s_{m+1}\in S_{m+1}\cap F\).
By definition of \(T^{d}\), there exists a sequence \(s_{1},...,s_{m+1}\) which is a succesful computation of \(\mathcal{A}\) on \(u\).\\
\begin{siderules}\color{blue}\textit{EXERCISE 2.3.5}\color{black}\\\\
\color{blue} Let \(\mathcal{A}_{1}\) and \(\mathcal{A}_{2}\) be deterministic finite automata. Consider the 
automata \(\mathcal{B}=\mathcal{A}_{1}\oplus\mathcal{A}_{1}\) and \(\mathcal{C}=\mathcal{A}_{1}\times \mathcal{A}_{1}\). Construct an automaton recognizing the
complements of \(L(\mathcal{B})\) and \(L(\mathcal{C})\) without using the subset construction.\\\\\color{black}
\(\mathcal{B}^{c}=\mathcal{A}_{1}^{c}\times\mathcal{A}_{2}^{c},\mathcal{C}=\mathcal{A}_{1}^{c}\oplus\mathcal{A}_{2}^{c}\)
\end{siderules}
\color{red}For every finite automaton there exists an automaton such that it accepts the complement of the language.\color{black}
\begin{siderules}\color{blue}\textit{EXERCISE 2.3.6}\color{black}\\\\
    \includegraphics{auto6}\\
\(S\) is finite and \(|S_{n}|\) is always increasing so there must be a point where \(S_{n+1}=S_{n}\).
\end{siderules}
\color{red}For every natural number \(n\le 1\) there exists an automaton with exactly \(n+1\)
states such that no deterministic automaton with less than \(2^{n}\) states recognizes the complement of \(L(\mathcal{A})\).\\\color{black}
Consider \(\Sigma=\{a,b\}\) and \(L=\{uav\;|\;u,v\in\Sigma^{*},|v|=n-1\}\). Let \(\mathcal{A}=(S,T,I,F)\) be the following:\\
\null\qquad 1. \(S=\{s_{0},...s_{n}\}\)\\
\null\qquad 2. \(T=\{(s_{0},\sigma,s_{0}),(s_{0},a,s_{1}),(s_{i},\sigma,s_{i+1})\}\)\\
\null\qquad 3. \(I=\{s_{0}\}\)\\
\null\qquad 4. \(F=\{s_{n}\}\)\\
\(\mathcal{A}\) recognizes \(L\). Now suppose there exists a deterministic automaton \(\mathcal{B}\) with less than \(2^{n}\) states that accept \(\Sigma^{*}\backslash L\).
Then there must be \(uav_{1}\) and \(ubv_{2}\) of length \(n\) that transforms the initial state to the same state.
Now \(uav_{1}u\in L\) and \(ubv_{2}u\notin L\). However the 2 words transforms the initial state to the same state. So no \(\mathcal{B}\) exists.\\
\begin{siderules}\color{blue}\textit{EXERCISE 2.3.7}\color{black}\\\\
\includegraphics{auto7}\\
\includegraphics{auto8}\\
Let \(\mathcal{B}\) be a deterministic automaton with less than \(p_{1}p_{2}...p_{n}\) states that recognizes \(L(\mathcal{A}_{n})\). Then there must be a cycle of length \(P<p_{1}p_{2}...p_{n}\) in \(\mathcal{B}\) if it follows the instruction \(aaa...\).
Therefore if instruction \(aaa...a\) of length \(x\) is not accepted, then \(x+P\) is also not accepted in \(\mathcal{B}\).
\(x\not\equiv 0\pmod{p_{i}}\) for all \(i\), and \(x+P\not\equiv 0\pmod{p_{i}}\) for all \(i\). If \(P\) is not divisible by \(p_{i}\), then there must exist a \(y\) where \(x+yP\equiv 0\pmod{p_{i}}\) because 
the set \(\{x\pmod{p_{i}},x+P\pmod{p_{i}},x+2P\pmod{p_{i}},...\}\) visits all elements. Therefore \(P\) is divisible by all \(p_{i}\), but \(P<p_{1}p_{2}...p_{n}\).
So no \(\mathcal{B}\) exist.
\end{siderules}
Suppose \(L_{1}\) and \(L_{2}\) are languages.
\[pref(L_{1},L_{2})=\{u\;|\;u\in L_{1} \text{ and for some string }w,uw\in L_{2}\}\]
\color{red}Let \(L_{1}\) and \(L_{2}\) be FA recognizable languages. Then \(pref(L_{1},L_{2})\) is FA recognizable.\\\color{black}
\(\mathcal{A}=\mathcal{A}_{1}\times \mathcal{A}_{2}\), except that \(F=\{(f,s)\;|\; f\in F_{1}\text{ and }G(s)\cap F_{2}\neq\varnothing\}\).\\\\
\[suff(L_{1},L_{2})=\{u\;|\;u\in L_{1} \text{ and for some string }w,wu\in L_{2}\}\]
\begin{siderules}\color{blue}\textit{EXERCISE 2.3.8}\color{black}\\\\
\color{red}Let \(L_{1}\) and \(L_{2}\) be FA recognizable languages. Then \(suff(L_{1},L_{2})\) is FA recognizable.\\\color{black}
\(\mathcal{A}=\mathcal{A}_{1}\times \mathcal{A}_{2}\), except that \(I=\{(i,s)\;|\;i\in I_{1}\text{ and }s\in G(I_{2})\}\).
\end{siderules}
\begin{siderules}\color{blue}\textit{EXERCISE 2.3.9}\color{black}\\\\
\includegraphics{auto9}\\
If \(L\) is FA recognizable, then it is obviously recognized by an NFA with silent moves because FA is a subset of NFA with silent moves.\\\\
If \(L\) is recognized by an NFA with silent moves \(\mathcal{A}=(S,T,I,F)\), then it is recognied by FA \(\mathcal{B}=(S,T_{B},I_{B},F)\) where 
\[T_{B}=\bigcup_{(s,\sigma,s')\in T_{1}}\{(s,\sigma,g)\;|\; g\in G_{T_{2}}(s')\}\text{ and }I_{B}=\bigcup_{i\in I}G_{T_{2}}(i)\]
\end{siderules}
Suppose our alphabet is of the form 
\[\Sigma_{1}\times \Sigma_{2}=\{(a,b)\;|\; a\in\Sigma_{1},b\in\Sigma_{2}\}\]
Any word of this alphabet is a sequence of pairs, which can be identified with 2 words each from \(\Sigma_{1}^{*}\) and \(\Sigma_{2}^{*}\).
\[pr_{1}(L)=\{u\in\Sigma_{1}^{*}\;|\;(u,v)\in L\text{ for some }v\in\Sigma_{2}^{*}\}\]
\[pr_{2}(L)=\{u\in\Sigma_{2}^{*}\;|\;(v,u)\in L\text{ for some }v\in\Sigma_{1}^{*}\}\]
\color{red}If language \(L\) over \(\Sigma=\Sigma_{1}\times \Sigma_{2}\) is FA recognizable, then so are the projections \(pr_{1}(L)\) and \(pr_{2}(L)\).\color{black}\\
\(\mathcal{A}_{1}=\mathcal{A}\), except \(S_{1}=\{(s,\sigma_{1},s')\;|\;(s,(\sigma_{1},\sigma_{2}),s')\in T\}\)
\subsection{The Myhill-Nerode Theorem}
2 inputs \(u\) and \(v\) are \textbf{in relation} \(\sim\) if for all states \(s\) and \(s'\) from \(S\), \(u\) transforms \(s\) to \(s'\) \(\Longleftrightarrow\) \(v\) transforms \(s\) to \(s'\).\\\\
\color{red}Let \(\mathcal{A}\) be an automaton. The relation \(\sim\) is an equivalence relation on \(\Sigma^{*}\). Moreover the index of the equivalence relation \(\sim\) is finite.\\\color{black}
It is reflexive, symmetric and transitive.\\
Take any input \(u\) and define \(f_{u}:S\to \mathcal{P}(S)\) to be \(f_{u}(s)=\{s'\;|\;\text{input }u\text{ transforms }s\text{ to }s'\}\).\\
For all \(u\) and \(v\) they are \(\sim\)-equivalent iff \(f_{u}=f_{v}\). Since the number of functions from \(S\) to \(\mathcal{P}(S)\) is finite, the index of \(\sim\) is finite.\\
\begin{siderules}\color{blue}\textit{EXERCISE 2.4.2}\color{black}\\\\
\includegraphics{auto10}\\
\(\{\varnothing\}, \{b, bb, bbb,...\}, \{a, ba, bba, ...\},\{\text{remaining that ends with }b\},\{\text{remaining that ends with }a\}\)
\end{siderules}
\textbf{right invariance}: if \(u\) is \(\sim\)-equivalent to \(v\), then \(uw\) is \(\sim\)-equivalent to \(vw\).\\\\
Therefore, if \(u\sim w\), then \(uw\in L(\mathcal{A})\) iff \(vw\in L(\mathcal{A})\).\\\\
\textbf{Myhill-Nerode equivalent relations}: Let \(L\) be a language. \(u\) and \(v\) are \(L\)-equivalent (\(u\sim_{L} v\)) if for all \(w\), \(uw\in L\) \(\Longleftrightarrow\) \(vw\in L\).\\\\
\color{red}Myhill-Nerode Theorem: A language \(L\) is FA recognizable iff \(\sim_{L}\) is of finite index.\color{black}\\
Suppose \(L=L(\mathcal{A})\). By above, \(\sim\) of \(\mathcal{A}\) is of finite index. And if \(u\sim v\) then \(u\sim_{L} v\). Therefore every \(\sim_{L}\) equivalence class is a union of \(\sim\)-equivalence classes. 
Hence \(\sim_{L}\) is of finite index. To see that \(L\) is a union of \(\sim_{L}\) equivalence classes note that if \(u\in L\) and \(v\sim_{L} u\), then \(v\in L\) because \(u\lambda, v\lambda\in L\).\\
Now suppose \(\sim_{L}\) is an equivalence relation of finite index. Let \(\mathcal{A}=(S,T,I,F)\) be the following:\\
\null\qquad \(S=\{[u]\;|\;u\in\Sigma^{*}\}\)\\
\null\qquad \(T=\{([u],\sigma,[v])\;|\;u\sigma\sim_{L}v\}\)\\
\null\qquad \(I=\{[\lambda]\}\)\\
\null\qquad \(F=\{[u]\;|\;u\in L\}\)\\
\(\mathcal{A}\) is a deterministic finite automaton. \(u\in L\) \(\Longleftrightarrow\) \([u]\in F\) \(\Longleftrightarrow\) \(u\in L(\mathcal{A})\)\\
\begin{siderules}\color{blue}\textit{EXERCISE 2.4.5}\color{black}\\\\
\color{blue}Give an example of an automaton \(\mathcal{A}\) for which \(\sim_{L(\mathcal{A})}\neq\sim\).\\\\
\includegraphics[width=10cm]{auto11}
\end{siderules}
A deterministic finite automaton \(\mathcal{A}\) is \textbf{minimal} if the number of states of every deterministic automaton recognizing \(L(\mathcal{A})\) is greater than or equal to the number of states of \(\mathcal{A}\).\\
\begin{siderules}\color{blue}\textit{EXERCISE 2.4.7}\color{black}\\\\
\color{blue}Let \(L\) be a FA recognizable language. Show that \(\mathcal{A}\) constructed in the proof of the Myhill-Nerode theorem is minimal.\\\\\color{black}
Let \(\mathcal{B}\) be a deterministic automaton with number of states less than number of equivalence classes of \(\sim_{L}\). Then, there must exist \(u,v\) s.t. \([u]\neq [v]\) but the final transformed state is the same.
This means that \(\exists w\;uw\in L,vw\notin L\) but this is not possible since it is deterministic and transforms to the same final state (either both are accepted or both are not accepted).
\end{siderules}
We say that automata \(\mathcal{A},\mathcal{B}\) over \(\Sigma\) are \textbf{isomorphic} if there exists a bijective \(h:S_{A}\to S_{B}\) s.t.:\\
\null\qquad \((s_{1},\sigma,s_{2})\in T_{A}\) \(\Longleftrightarrow\) \((h(s_{1}),\sigma,h(s_{2}))\in T_{B}\)\\
\null\qquad \(s\in I_{A}\) \(\Longleftrightarrow\) \(h(s)\in I_{B}\)\\
\null\qquad \(s\in F_{A}\) \(\Longleftrightarrow\) \(h(s)\in F_{B}\)\\
\begin{siderules}\color{blue}\textit{EXERCISE 2.4.8}\color{black}\\\\
\color{blue}Two minimal automata recognizing the same language are isomorphic.\\\\\color{black}
\(\mathcal{A}\) and \(\mathcal{B}\) are isomorphic \(\Longleftrightarrow\) (\(\forall u,v\;(u,v\text{ go to same state in }\mathcal{A}\Longleftrightarrow u,v\text{ go to same state in }\mathcal{B})\) and \(F_{A}=F_{B}\))\\
\(\mathcal{A}\) and \(\mathcal{B}\) are not isomorphic \(\Longleftrightarrow\) (\(\exists u,v\;(u,v\text{ go to same state in }\mathcal{A}\Longleftrightarrow u,v\text{ go to diff state in }\mathcal{B})\) or \(F_{A}\neq F_{B}\))\\
Suppose there exists non-isomorphic minimal automata \(\mathcal{A}\) and \(\mathcal{B}\). Assume first condition is true. Since \(u,v\) go to the same state in \(\mathcal{A}\), they are \(\sim_{L}\)-equivalent. This means 2 \(\sim_{L}\)-equivalent strings go to different states in \(\mathcal{B}\).
Construct new automata \(\mathcal{B}_{2}=(S,T,I,F)\):\\
Denote the state \(u\) goes to as \(s_{u}\) and the state \(v\) goes to as \(s_{v}\).\\
\null\qquad \(S=S_{B}\backslash\{s_{v}\}\)\\
\null\qquad \(T=T_{B}\) but every arrow that points to \(s_{v}\) now points to \(s_{u}\)\\
\null\qquad \(I=I_{B}\)\\
\null\qquad \(F=F_{B}\) without \(s_{v}\) if its in (\(s_{u}\in F_{B}\Longleftrightarrow s_{v}\in F_{B}\))\\
We have constructed an automaton that recognizes the same language with number of states less than that of the minimal automaton. Therefore first condition cannot hold.
It is clear that if the first condition does not hold and \(F_{A}\neq F_{B}\) then they dont recognize the same language. Therefore it is impossible that \(\mathcal{A}\) and \(\mathcal{B}\) are not isomorphic.
\end{siderules}
\subsection{The Kleene Theorem}
The \textbf{trivial languages} over \(\Sigma\) are \(\varnothing,\{\lambda\},\{a_{1}\},...,\{a_{n}\}\).\\\\
The concatenation of \(L_{1}\) and \(L_{2}\) is \(L_{1}\cdot L_{2}=\{uv\;|\;u\in L_{1},v\in L_{2}\}\). The concatenation operation makes the set of all languages a monoid, and is associative:
\[(L_{1}\cdot L_{2})\cdot L_{3}=L_{1}\cdot (L_{2}\cdot L_{3})\]
Another operation is union: \(L_{1}+L_{2}=L_{1}\cup L_{2}\)\\\\
The third operation is the \textbf{Kleene star}: \(L^{*}=\{\lambda\}+L+(L\cdot L)+(L\cdot L\cdot L)+...=L^{0}+L^{1}+L^{2}+L^{3}+...\)\\\\
That is why the notation \(\Sigma^{*}\) is the way it is.\\\\
\textit{Stage 0. }\(R_{0}\) is the class of trivial languages.\\
\textit{Stage \(t+1\). } Suppose \(R_{t}=\{L_{1},L_{2},...,L_{k}\}\) has been defined. Let \(R_{t+1}\) be 
\[R_{t}\cup\{L_{i}^{*}\;|\;i=1,...,k\}\cup\{L_{i}\cdot L_{j}\;|\;i,j=1,...,k\}\cup\{L_{1}+L_{j}\;|\;i,j=1,...,k\}\]
Define \(R\):\[R=\bigcup_{t\in\omega}R_{t}\]
A language \(L\subseteq\Sigma^{*}\) is \textbf{regular} if \(L\) is in \(R\).
\begin{siderules}\color{blue}\textit{EXERCISE 2.5.2}\color{black}\\\\
\color{blue}Let \(\Sigma=\{a,b\}\). Show that the following languages are regular.\\
\(\{ab^{n}\;|\;n\in\omega\}\) \color{black} \(\{a\}\cdot \{b\}^{*}\)\\\color{blue}
\(\{(ab)^{2n}\;|\;n\in\omega\}\) \color{black} \((\{a\}\cdot\{b\}\cdot\{a\}\cdot\{b\})^{*}\)\\\color{blue}
any finite language \color{black} Use only \(\cdot\) and \(+\) \\\color{blue}
any cofinite language \color{black} Use only \(\cdot\) and \(+\), then \(\cdot (\{a\}+\{b\})^{*}\)\\\color{blue}
\(\{uabv\;|\;u,w\in\Sigma^{*}\}\) \color{black} \((\{a\}+\{b\})^{*}\cdot\{a\}\cdot\{b\}\cdot(\{a\}+\{b\})^{*}\)\\\color{blue}
\(\{aub\;|\;u\in\Sigma^{*}\}\) \color{black} \(\{a\}\cdot (\{a\}+\{b\})^{*}\cdot \{b\}\)
\end{siderules}
Name the regular languages \(\varnothing, \{\lambda\}, \{a_{1}\},...,\{a_{n}\}\) with \textbf{symbols} \(\mathbf{e},\mathbf{0},\mathbf{a_{1}},...,\mathbf{a_{n}}\). 
\textbf{Regular expressions} represent regular languages and are special type of words over the new alphabet: \(\mathbf{e},\mathbf{0},\mathbf{a_{1}},...,\mathbf{a_{n}},+,*,\cdot,(,)\)\\\\
The \textbf{atomic regular expressions} are \(\mathbf{e},\mathbf{0},\mathbf{a_{1}},...,\mathbf{a_{n}}\). \(L(\textbf{e})=\varnothing,L(\textbf{0})=\{\lambda\},L(\mathbf{a_{i}})=\{a_{i}\}\)\\
Suppose regular expressions \(\textbf{r}_{1},\textbf{r}_2\) and \(L(\textbf{r}_{1}),L(\textbf{r}_{2})\) are defined. Then \((\textbf{r}_{1}\cdot \textbf{r}_{2}),(\textbf{r}_{1}+\textbf{r}_{2}),(\textbf{r})^{*}\) are \textbf{regular expressions}. 
The languages defined by these expressions are obvious. operator precedance: \(*<\cdot<+\); we denote the set of all regular expressions \textbf{RE}.\\\\
\color{red}A language \(L\) is regular iff \(L=L(\textbf{r})\) for some regular expression \textbf{r}.\color{black}\\
For forward, show every language in every \(R_{t}\) can be represented by a regular expression by induction; for backward, show \(L\) is regular using the definition of regular expressions.\\\\
\color{red}Kleene Theorem: A language \(L\) is regular iff \(L\) is FA recognizable.\\\color{black}

\end{document}
