\documentclass{article}
\usepackage[utf8]{inputenc}
\usepackage{amsmath}
\usepackage{amssymb}
\usepackage[dvipsnames]{xcolor}
\usepackage[a4paper, total={6in, 10in}]{geometry}
\usepackage{tcolorbox}
\usepackage{mdframed}
\usepackage[hidelinks]{hyperref}
\usepackage{amsfonts}
\usepackage{mathrsfs}
\usepackage{centernot}
\usepackage[percent]{overpic}

\setlength{\jot}{10pt}

\newmdenv[
  topline=false,
  bottomline=false,
  skipabove=\topsep,
  skipbelow=\topsep,
  leftmargin=-10pt,
  rightmargin=-10pt,
  innertopmargin=0pt,
  innerbottommargin=0pt,
  linecolor=blue
]{siderules}

\title{Homotopical Topology\\Notes}
\author{toshinari tong}
\begin{document}
\maketitle
\section{Classical Spaces}
\subsection{Euclidean Spaces, Spheres, and Balls}
\(\mathbb{R}^{n}, \mathbb{C}^{n}\), sphere \(S^{n}\), ball \(D^{n}\); \(\mathbb{R^{\infty}}\) is the inductive limit of the 
chain \(\mathbb{R}^{1}\subset\mathbb{R}^{2}\subset\mathbb{R}^{3}\subset...\); thus \(\mathbb{R}^{\infty}\) is the set of 
sequences of real numbers with only finitely many nonzero terms. The topology in \(\mathbb{R}^{\infty}\) is introduced by the 
rule: A set \(F\subset\mathbb{R}^{\infty}\) is closed \(\Longleftrightarrow\) all intersections \(F\cap\mathbb{R}^{n}\) are 
closed in respective spaces. \(\mathbb{C}^{\infty},S^{\infty},D^{\infty}\) have a similar sense.
\begin{siderules}\color{blue}Exercise 1. Show that sequence \((a_{1},0,0,...),(0,a_{2},0,...),(0,0,a_{3},...),...\) 
has a limit \(\Longleftrightarrow\) it has finitely many nonzero terms.\color{black}
\\\\
Since in each dimension it is nonzero at most once the limit point if it exists must be \((0, 0, 0,...)\). Consider open 
neighbourhood \(((-|a_{1}/2|, |a_{1}/2|), ((-|a_{2}/2|, |a_{2}/2|),(-|a_{3}/2|,|a_{3}/2|),...)\) (if \(a_{i} =0\) then take 
\((-1,1)\)): If \(a_{i}\) is nonzero then it is outside this neighbourhood. But there must \(\exists N\) s.t. \(\forall n>N\) 
the \(n\)-th point is in this neighbourhood, or in other words is zero. Therefore it only has finitely many nonzero terms.
\end{siderules}
\end{document}
